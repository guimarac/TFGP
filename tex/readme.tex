\documentclass[a4paper,oneside]{memoir}

\usepackage[utf8]{inputenc}
\usepackage[T1]{fontenc}

\usepackage{lmodern}

%\usepackage[czech]{babel}
\usepackage{amsmath,amssymb,mathtools,amsthm,bm}

\usepackage{xspace}


\usepackage{lipsum}

\usepackage{hyperref}
\hypersetup{
    colorlinks,
    citecolor=black,
    filecolor=black,
    linkcolor=black,
    urlcolor=black
}

\title{TFGP readme}

\author{Tomáš Křen}

\hyphenation{vě-dec-ká}

\begin{document}

\theoremstyle{plain} 
\newtheorem{theorem}{Theorem} 
\newtheorem{proposition}{Proposition} 
\newtheorem{lemma}{Lemma} 
\newtheorem{preLemma}{Pre-Lemma} 
\newtheorem*{corollary}{Corollary}

\theoremstyle{definition} 
\newtheorem*{definition}{Definition} 
\newtheorem*{preDefinition}{Pre-Definition} 
\newtheorem{conjecture}{Conjecture}
\newtheorem*{example}{Example} 

\theoremstyle{remark} 
\newtheorem*{remark}{Remark} 
\newtheorem*{note}{Note} 
\newtheorem{case}{Case}

\frontmatter
\mainmatter
\maketitle

%\renewcommand{\chaptername}{Akt}

\tableofcontents*
%\clearpage

\chapter{Generating}

\section{General notions}

\begin{definition}
A $\mathit{term:type}$ statement $\mathit{M}:\mathit{\tau}$ states that (program) term $M$ has type $\tau$.   
A \textit{declaration} is a statement $s : \tau$ where $s$ is a term symbol and $\tau$ is a type.
Often we will write $s : \tau_s$ to emphasize that $\tau_s$ is the type of symbol $s$ in the supposed context.
A \textit{context} is set of declarations with distinct term symbols.\footnote{Interestingly, the definition of a \textit{context} and definition of a \textit{substitution} are almost the same. The difference is that "keys" in a context are term symbols/variables, whereas substitution "keys" are type variables. Maybe this fact could be utilized in an interesting way...}
\end{definition}

\newcommand{\then}{\Rightarrow}
\newcommand{\E}[2]{(\exists #1)\ #2}
\newcommand{\A}[2]{(\forall #1)\ #2}
\newcommand{\Ain}[3]{(\forall #1 \in #2)\ #3}


\newcommand{\op}{\operatorname}

\newcommand{\ar}{\rightarrow}
\newcommand{\ap}[2]{(#1\,#2)}
\newcommand{\defi}{\coloneqq}
\newcommand{\defe}{\mathrel{\vcentcolon\equiv}}

\newcommand{\binRule}[3]{\dfrac{#1\ ,\ #2}{#3}}
\newcommand{\triRule}[4]{\dfrac{#1\ ,\ #2\ , \ #3}{#4}}
\newcommand{\isSub}[1]{#1\ \mathit{substitution}}
\newcommand{\MGU}[2]{\op{MGU}(#1,#2)}

\newcommand{\subAx}{\textit{SUB-AX}\xspace}
\newcommand{\mguMp}{\textit{MGU-MP}\xspace}
\newcommand{\abs}[1]{\lvert #1 \rvert}

Suppose we have a context $\Gamma$. Let us consider $\mathit{term:type}$ statements derivable from the following rules (let us call them \subAx and \mguMp):

~

$\binRule{(s,\tau_s) \in \Gamma}{\isSub{\sigma}}{s : \sigma(\tau_s)}$
~~~
$\triRule{F : \tau_1 \ar \tau_2}{X : \tau^\prime_1}{\sigma = \MGU{\tau_1}{\tau^\prime_1}}{\ap{F}{X} : \sigma(\tau_2)}$


\begin{definition}
Let $M$ be a term. Term size $\abs{M}$ is the number of symbols in $M$; e.g. $\abs{\ap{f}{\ap{\ap{g}{h}}{f}}} = 4$. 
\end{definition}

\section{Reusable generating}

\newcommand{\inhab}[1]{\op{I}(#1)}
\newcommand{\sigmaPr}{\sigma^\prime}
\newcommand{\tauPr}{\tau^\prime}
\newcommand{\xPr}{x^\prime}
\newcommand{\subOrd}[2]{#1 \preccurlyeq #2}
\newcommand{\strictSubOrd}[2]{#1 \prec #2}



\begin{align*}
\op{I}_\tau(\sigma)   &\defe \E{M}{M : \sigma(\tau)}  \\
\subOrd{\tauPr}{\tau} &\defe \E{\sigma}{\tauPr = \sigma(\tau)} \\
\strictSubOrd{\tauPr}{\tau} &\defe \subOrd{\tauPr}{\tau} \wedge \neg (\subOrd{\tau}{\tauPr}) \\
\tauPr \nsucc \tau &\ \equiv \subOrd{\tauPr}{\tau} \vee  \neg ( \subOrd{\tau}{\tauPr})   \\   
\op{MGI}_\tau(\sigma) &\defe \op{I}_\tau(\sigma) \wedge \Ain{\sigmaPr}{\op{I}_\tau}{\sigmaPr(\tau) \nsucc \sigma(\tau)}\\
~\\
\op{I}_\tau &\defi \{ \sigma \mid \op{I}_\tau(\sigma)  \}  \\
 &= \{ \sigma \mid \E{M}{M : \sigma(\tau)}  \} \\
\op{MGI}_\tau &\defi \{ \sigma \mid \op{MGI}_\tau(\sigma)  \}  \\
 &= \{ \sigma \mid \op{I}_\tau(\sigma)\ ,\ \Ain{\sigmaPr}{\op{I}_\tau}{\subOrd{\sigmaPr(\tau)}{ \sigma(\tau)}}  \} \\
 &= \{ \sigma \mid \sigma \in \op{I}_\tau,\Ain{\sigmaPr}{\op{I}_\tau}{\subOrd{\sigmaPr(\tau)}{ \sigma(\tau)}}  \}
\end{align*}

I is for Inhabitator. MG is for Most General. In general the Most General relation can be stated as:

$$\op{MGx}(X,\preceq) \defi \{ x \mid x \in X, \Ain{\xPr}{X}{\xPr \preceq x} \}$$



\begin{align*}
\tauPr \nsucc \tau &\equiv \neg (\strictSubOrd{\tau}{\tauPr})   \\
  &\equiv \neg ( \subOrd{\tau}{\tauPr} \wedge \neg (\subOrd{\tauPr}{\tau}) )   \\
  &\equiv \neg ( \subOrd{\tau}{\tauPr}) \vee \subOrd{\tauPr}{\tau}   \\  
  &\equiv \subOrd{\tauPr}{\tau} \vee  \neg ( \subOrd{\tau}{\tauPr})    
\end{align*}

~

~

\newcommand{\subs}[2]{\op{subs}_{#2}(#1)}

\begin{preDefinition}
$ \subs{\tau_g}{n} \defi \{ \sigma_g \mid \E{M}{M : \sigma_g(\tau_g)}, \abs{M} = n \}$
\end{preDefinition}

Problem s touhle starou definicí: asi zahrnuje i zbytečně konkrétní substituce.

Kdybych měl $\Gamma = \{id : \alpha \ar \alpha \}$

Tak $\{\alpha \mapsto Int, \beta \mapsto  Int\} \in \subs{\alpha \ar \beta}{1}$

Čili chceme něco jako Most General Subs který pokrejvá všecky Mka.. 



~










$ \op{weakSubs}(\tau_g) \defi \{ \sigma_g \mid \inhab{\sigma_g(\tau_g)} \}$


\begin{preLemma} 
$ \subs{\tau_g}{1} = \{ \sigma_g \mid (s,\tau_s) \in \Gamma, \sigma_g = \MGU{\tau_g}{\tau^\mathit{fresh}_s}  \} $
\end{preLemma} 




\backmatter
\end{document}

