\documentclass[a4paper,oneside]{memoir}

\usepackage[utf8]{inputenc}
\usepackage[T1]{fontenc}

\usepackage{lmodern}

%\usepackage[czech]{babel}
\usepackage{amsmath,amssymb,mathtools,amsthm,bm}

\usepackage{xspace}


\usepackage{lipsum}

\usepackage{hyperref}
\hypersetup{
    colorlinks,
    citecolor=black,
    filecolor=black,
    linkcolor=black,
    urlcolor=black
}

%\usepackage{enumitem}

\title{TFGP readme}

\author{Tomáš Křen}

\hyphenation{vě-dec-ká}

\begin{document}

\theoremstyle{plain} 
\newtheorem{theorem}{Theorem} 
\newtheorem{proposition}{Proposition} 
\newtheorem{lemma}{Lemma} 
\newtheorem{preLemma}{Pre-Lemma} 
\newtheorem*{corollary}{Corollary}

\theoremstyle{definition} 
\newtheorem*{definition}{Definition} 
\newtheorem*{preDefinition}{Pre-Definition} 
\newtheorem{conjecture}{Conjecture}
\newtheorem*{example}{Example} 

\theoremstyle{remark} 
\newtheorem*{remark}{Remark} 
\newtheorem*{note}{Note} 
\newtheorem{case}{Case}

\frontmatter
\mainmatter
\maketitle

%\renewcommand{\chaptername}{Akt}

\tableofcontents*
%\clearpage

\newcommand{\red}[1]{{\color{red} #1}}

\chapter{i já it}

\section{Osnova}

\textbf{Jak to napsat?}

- Jako bych psal TFGP blogísek, todle je trénink.

- Tak aby se to z toho pěkně pochopilo, ne aby to bylo HC formální nečitelná píčovina.

~\\

\subsection{Halabala témata}

\textbf{Problém popisu dagů} : 

- workflows jsou dagy

- dagy jde popsat operacema skladající jiný dagy

\textbf{Parametricky polymorfní Typovej systém} : 

- chceme konstruovat jen korektní dagy

- chceme aby na sebe navazovali správně počty atd

\textbf{Aplikativní notace}

- klasicka představa je že funkce jsou ve vnitřních uzlech a jejich parametry jsou syni

- aplikativní notace využívá toho, že každou funkci si díky curringu mužu chápat jako fci 1 proměný. 

- pak mužu vzit stromovou reprezentaci explicitně zachycující jednotlivé aplikace

\textbf{Generování}

Dosud jsme popsali co je cílem tvořit, ted popíšem jak na to jdeme.

- 1. základ našeho přístupu: generuje strom pro danou velikost stromu (počet symbolů).
  
  - což mimojiné umožnuje mnohem přímější kontrolu nad tím, z jakého rozložení taháme naše jhedince 
  - zde popsat to jak to děláme (32, 16,16, 8,8,8,8, ...)

- 2. základ našeho přístupu: počítáme si počty stromů pro jednotlivý dotazy 
     - abychom byli schopný generovat (semi-)uniformě

- 3. základ našeho přístupu: ex. sigma že z gamma de vyvodit že M je typu sigma(tau)

  - k tomu je potřeba definovat prerekvizitní pojmy, minimálně tyto:
    - substituce
    - a to jakym způsobem vypadá dotaz (k,tau) - a že je teda obecnejší to tau než přímo typ že pak M:Tau
      - to by šlo snad názorně ukázat na příkladě generování stromu k>1:
      - Generujem Dag D LD, k>1 -> v kořeni stromu je aplikace tedy:
         - levej syn je funkce z něčeho do (Dag D LD), pravej syn je to něco

  - to kulminuje v hodící se funkci subs, která pro danej dotaz (k,tau) vrátí substituce spolu s počtama stromů


- \textit{vygenerování jednoho jedince/stromu/programu}

 - (a) strom velikosti 1 - výběr symbolu z gammy aby to pasovalo
 - (b) strom velikosti k>1 - tedy jde o aplikaci a

- \textit{předpočítání pomocných dat pro generování}


- "semi-unfiromní" generování - poznámka o nedokonalosti v uniformitě

~\\

~\\

\textbf{Obecnej výhled osnovy}:

- Čim začít: 
Co je to ml workflow. 
ML workflows přirozeně tvoří graf. 
Natrénovat a vyhodnotit takový ensamble je to výpočetně náročný a tak chceme vyloučit cykly. 
Tedy budeme tvořit dagy. 

Architektura našeho systému sestává ze dvou základních úkolů:
(1) Vytvořit DAG reprezentující daný workflow - pomocí evoluce by GP
(2) Natrénovat a vyhodnotit daný workflow

Během evoluce však nepracujemes přímo s grafovou reprezentací ale nepřímo skrze stromovou reprezentaci programu, jehož výsledkem je DAG popisující daný workflow.

Tedy jsou to 3 kroky: strom popisuje program který když se vyhodnotí tak jeho výsledek je graf podle kterého se postavý ML workflow který se natrénuje a vyhodnotí.


Základní stavební bloky jsou basic klasifikační metody.
Dálešími elementárními prvky jsou jedotlivé preprocessingy, clusteringy a votovací metody. 
Každou z elementárních metod můžeme chápat jako dag.
Tyto elementární prvky jsou kombinovány do větších pomocí operací skladajících dva a víc menších dagů do jednoho většího.

 


Výsledkem prvního kroku je tedy graf. 
To jak zacházíme s Při vytváření daného , ten však může být výsledkem  

My ale
Dagy reprezentujeme jako výrazy.

- Co je jádro:

- Jak to schrnout:

~\\


\section{Individual representation}

\section{Generating}



\chapter{Gecco}

\textbf{Uniform Generating for Strongly Typed Genetic Programming with Parametric Polymorphism}

In this paper we present a novel tree generating method for strongly typed genetic programming using a type system with parametric polymorphism. Our method is capable of uniform generating suitable both for a population initialization and a mutation operation. In order to perform this task effectively, the method utilizes type normalization and caching using dynamic programming. We concentrate on a deeper description of theoretical and technical aspects of our method to show its connection with logic programming. The method is demonstrated, analyzed and experimentally evaluated on a simple problem.


\chapter{Generating}

\section{General notions}

\begin{definition}
A $\mathit{term:type}$ statement $\mathit{M}:\mathit{\tau}$ states that (program) term $M$ has type $\tau$.   
A \textit{declaration} is a statement $s : \tau$ where $s$ is a term symbol and $\tau$ is a type.
Often we will write $s : \tau_s$ to emphasize that $\tau_s$ is the type of symbol $s$ in the supposed context.
A \textit{context} is set of declarations with distinct term symbols.\footnote{Interestingly, the definition of a \textit{context} and definition of a \textit{substitution} are almost the same. The difference is that "keys" in a context are term symbols/variables, whereas substitution "keys" are type variables. Maybe this fact could be utilized in an interesting way...}
\end{definition}

\newcommand{\sigmaPr}{\sigma^\prime}
\newcommand{\tauPr}{\tau^\prime}
\newcommand{\xPr}{x^\prime}
\newcommand{\nPr}{n^\prime}
\newcommand{\nPrr}{n^{\prime\prime}}
\newcommand{\nPrrr}{n^{\prime\prime\prime}}
\newcommand{\tausPr}{\tau_s^\prime}
\newcommand{\s}{\sigma}
\newcommand{\Th}{\theta}
\newcommand{\sPr}{\sigmaPr}
\newcommand{\thPr}{\theta^\prime}



\newcommand{\then}{\Rightarrow}
\newcommand{\E}[2]{(\exists #1)\ #2}
\newcommand{\A}[2]{(\forall #1)\ #2}
\newcommand{\Ain}[3]{(\forall #1 \in #2)\ #3}


\newcommand{\op}{\operatorname}

\newcommand{\ar}{\rightarrow}
\newcommand{\ap}[2]{(#1\,#2)}
\newcommand{\defi}{\coloneqq}
\newcommand{\defe}{\mathrel{\vcentcolon\equiv}}

\newcommand{\binRule}[3]{\dfrac{#1\ ,\ #2}{#3}}
\newcommand{\triRule}[4]{\dfrac{#1\ ,\ #2\ , \ #3}{#4}}
\newcommand{\isSub}[1]{#1\ \mathit{substitution}}
\newcommand{\MGU}[2]{\op{MGU}(#1,#2)}
\newcommand{\mgu}[1]{\op{MGU}(#1)}

\newcommand{\AX}{\textit{AX}\xspace}
\newcommand{\subAx}{\textit{SUB-AX}\xspace}
\newcommand{\mguMp}{\textit{MGU-MP}\xspace}
\newcommand{\abs}[1]{\lvert #1 \rvert}

Suppose we have a context $\Gamma$. Let us consider $\mathit{term:type}$ statements derivable from the following rules (let us call them \subAx and \mguMp):

~

$\binRule{(s,\tau_s) \in \Gamma}{\isSub{\sigma}}{s : \sigma(\tau_s)}$
~~~
$\triRule{F : \tau_1 \ar \tau_2}{X : \tau^\prime_1}{\sigma \in \MGU{\tau_1}{\tau^\prime_1}}{\ap{F}{X} : \sigma(\tau_2)}$


\begin{definition}
Let $M$ be a term. Term size $\abs{M}$ is the number of symbols in $M$; e.g. $\abs{\ap{f}{\ap{\ap{g}{h}}{f}}} = 4$. 
\end{definition}

\newcommand{\inhab}[1]{\op{I}(#1)}

\newcommand{\tord}{\preccurlyeq}
\newcommand{\stord}{\prec}
\newcommand{\ordt}{\tord_\tau}
\newcommand{\tek}{\sim}
\newcommand{\ntek}{\nsim}
\newcommand{\ekt}{\tek_\tau}
\newcommand{\nekt}{\ntek_\tau}
\newcommand{\nsucct}{\nsucc_\tau}

\newcommand{\MGI}[1]{\op{MGI}(#1)}
\newcommand{\MGIt}{\MGI{\tau}}
\newcommand{\It}{\op{I}(\tau)}

\newcommand{\ids}{\sigma_{\op{id}}}

\newcommand{\U}[2]{\op{U}(#1,#2)}
\newcommand{\Utt}{\U{\tau}{\tauPr}}
\newcommand{\MGUtt}{\MGU{\tau}{\tauPr}}

\newcommand{\e}[2]{\op{E}_{#1}(#2)}
\newcommand{\restrict}[2]{{#1}_{\mid #2}}
\newcommand{\fresh}[2]{\op{fresh}_{#1}(#2)}
\newcommand{\newVar}[1]{\op{newVar}(#1)}
\newcommand{\Ss}[1]{\op{ss}(#1)}
\newcommand{\TS}[2]{\op{ts}_{#1}(#2)}
\newcommand{\ts}[2]{\op{ts}_{#1}(#2)}
\newcommand{\TSij}[3]{\op{ts}_{#1,#2}(#3)}
\newcommand{\trees}[2]{\op{trees}_{#1}(#2)}
\newcommand{\FX}{\ap{F}{X}}
\newcommand{\sF}{\s_{F}}
\newcommand{\sX}{\s_{X}}
\newcommand{\vars}[1]{\op{vars}(#1)}
\newcommand{\dom}[1]{\op{dom}(#1)}
\newcommand{\IH}{induction hypothesis\xspace}
\newcommand{\discup}{~\mathbin{\dot{\cup}}~}



\section{Reusable generating}

\begin{definition}
\begin{align*}
\e{k}{\tau} \defi& \{ M \mid \abs{M} = k, \E{\s}{ M : \s(\tau) } \} \\
\ts{1}{\tau,n} \defi&  \{ (s,\restrict{\mu}{\tau}, \nPr) \mid \\
 & ~~ (s,\tau_s) \in \Gamma, \\
 & ~~ (\tausPr,\nPr) = \fresh{\tau}{\tau_s, n}, \\
 & ~~ \mu = \mgu{\tau,\tausPr}
\} \\
\ts{i,j}{\tau,n} \defi& \{ (\FX, \restrict{(\sX \circ \sF)}{\tau}, \nPrrr) \mid \\ 
  & ~~ (\alpha, \nPr) = \newVar{\tau,n}, \\
  & ~~ (F,\sF,\nPrr) = \ts{i}{\alpha \ar \tau,\nPr}, \\
  & ~~ (X,\sX,\nPrrr) = \ts{j}{\sF(\alpha), \nPrr} 
\} \\
\ts{k > 1}{\tau,n} \defi& \bigcup\limits_{i=1}^{k-1}  \TSij{i}{k-i}{\tau, n}
\end{align*}
\end{definition}

\begin{lemma}[Core lemma about $\op{ts}_k$]
Let $(M, \s, \nPr) \in \ts{k}{\tau,n}$, then: 
\begin{align}
& \abs{M} = k,  
&\textit{(size)} \label{tsSize} \\
& M : \s(\tau), 
&\textit{(correctness)} \label{tsCorrectness} \\
& \A{\sPr}{M : \sPr(\tau) \then  \E{\theta}{\sPr(\tau) = \theta\circ\s(\tau)}},
& \textit{(generality)} \label{tsGenerality} \\
& \dom{\s} \subseteq \vars{\tau},\op{OK}(\tau,\nPr,\s), \nPr \geq n.
& \textit{(technical)} \label{tsTechnical}
\end{align}
\end{lemma}
\begin{proof}

By induction on $k$. 

Let $k = 1$. 
Let $(M, \s, \nPr) \in \ts{1}{\tau,n}$, 
then $M = s, \s = \restrict{\mu}{\tau}$,

where 
$(s,\tau_s) \in \Gamma,
(\tausPr,\nPr) = \fresh{\tau}{\tau_s, n},
\mu = \mgu{\tau,\tausPr}$.
 
(\ref{tsSize}) $\abs{M} = \abs{s} = 1$, since $s$ is a symbol of $\Gamma$.

(\ref{tsCorrectness}) From \textit{lemma about fresh$_\tau$} we have that 
$\E{\rho}{\rho(\tau_s) = \tausPr}$. 

Let us use (\AX) rule:
$\binRule{(s,\tau_s) \in \Gamma}{\isSub{\mu \circ \rho}}
{s : \mu(\rho(\tau_s))}$
\red{todo střízlivost}

Since $\mu = \mgu{\tau,\tausPr}$, we have $\mu(\tausPr) = \mu(\tau)$, 

therefore $\mu(\rho(\tau_s)) = \mu(\tausPr) = \mu(\tau) = \restrict{\mu}{\tau}(\tau)$, 

thus $s : \restrict{\mu}{\tau}(\tau)$.

(\ref{tsGenerality}) Let $\sPr$ be a substitution such that $s : \sPr(\tau)$.

From \textit{factoring lemma} we have that
$\E{\nu}{\nu(\tau) = \nu(\tausPr) = \sPr(\tau)}$. 

But $\mu = \mgu{\tau,\tausPr}$, thus 
$\mu$ is more general unification of $\tau$ and $\tausPr$ than $\nu$.

Therefore $\E{\theta}{\nu = \theta \circ \mu}$.

Finally $\sPr(\tau) = \nu(\tau) = \theta \circ \mu(\tau) = \theta \circ \restrict{\mu}{\tau} (\tau)$.

(\ref{tsTechnical}) Trivially $\dom{\restrict{\mu}{\tau}} \subseteq \vars{\tau}$.

\red{ todo zbytek technikálií (na papíře \#freshSummary)}.

Let $k > 1$. 

(\ref{tsSize})

(\ref{tsCorrectness})

(\ref{tsGenerality})

(\ref{tsTechnical})


\end{proof}


~\\

~\\

\section{Reusable generating - older approach}


\begin{definition}
\begin{align*}
\e{k}{\tau} &\defi \{ M \mid \abs{M} = k, \E{\s}{ M : \s(\tau) } \}   \\
\Ss{\tau}   &\defi \{ (s,\restrict{\mu}{\tau}) \mid (s,\tau_s) \in \Gamma, \mu = \MGU{\tau}{\fresh{\tau}{\tau_s}}  \} \\
\TSij{i}{j}{\tau} &\defi \{ \FX,\restrict{(\sX \circ \sF)}{\tau}) \mid \alpha = \newVar{\tau}, \\
  & ~~~~~~~~~~~~~~~~~~~~~~~~~~~~~~~~~~ (F,\sF) = \TS{i}{\alpha \ar \tau}, \\
  & ~~~~~~~~~~~~~~~~~~~~~~~~~~~~~~~~~~ (X,\sX) = \TS{j}{\sF(\alpha)} \} \\
\TS{k}{\tau} &\defi
\begin{cases*}
  \Ss{\tau} & $k = 1$  \\
  \bigcup\limits_{i=1}^{k-1}  \TSij{i}{k-i}{\tau}  & $k > 1$
\end{cases*}\\
\trees{k}{\tau} &\defi \{ M \mid (M,\_) \in \TS{k}{\tau} \}
\end{align*}
\end{definition}

\begin{proposition}
\begin{equation}\label{eq:treesEqE}
\trees{k}{\tau} = \e{k}{\tau}
\end{equation}
\begin{equation}\label{eq:mostGenSub}
(M,\s) \in \TS{k}{\tau} \then (M : \s(\tau) \wedge \A{\sPr}{(M : \sPr(\tau) \then \sPr(\tau) \tord \s(\tau))})
\end{equation}

\begin{proof}
First we show that $\trees{k}{\tau} \subseteq \e{k}{\tau}$ by induction on $k$ (size of term).
Together with it we also show the proposition \ref{eq:mostGenSub}.

Let $k = 1$. 

Let $M \in \trees{1}{\tau}$, we shall show that $M \in \e{1}{\tau}$.

$(M,\mu) \in \Ss{\tau}$, thus $M = s, (s,\tau_s) \in \Gamma, \mu = \MGU{\tau}{\tauPr_s}$,

where $\tauPr_s = \fresh{\tau}{\tau_s} = \rho(\tau_s)$, where $\rho$ is \textit{renaming} 
such that $\vars{\tau} \cap \vars{\tau_s} = \emptyset$.

To show that $s \in \e{1}{\tau}$ we shall show that 

$\abs{s} = 1$ and 

find $\s$ such that $ s : \s(\tau)$. 

Since s is symbol we have $\abs{s} = 1$.

\red{TODO Diagram konstrukce $\mu \circ \rho$}

In order to produce term : type statement with symbol as term, we need to use \subAx rule:

~

$\binRule{(s,\tau_s) \in \Gamma}{\isSub{\mu \circ \rho}}{s : \mu(\rho(\tau_s))}$

~

Since $\mu(\rho(\tau_s)) = \mu(\tau_s) = \mu(\tau)$, we have $s : \mu(\tau)$.
Thus the desired $\s = \mu$.

Now we follow with proof of proposition \ref{eq:mostGenSub} for $k = 1$.
Since $\sigma = \mu$ we have the first part of the consequent $s : \mu(\tau)$. 
For the second part let us assume that we have substitution $\sPr$ such that 
statement $s : \sPr(\tau)$ holds. This statement had to be produced by \subAx rule:

~

$\binRule{(s,\tau_s) \in \Gamma}{\isSub{\theta}}{s : \theta(\tau_s)}$

~

where $\theta(\tau_s) = \sPr(\tau)$.
 
Our approach to showing that $\sPr(\tau) \tord \mu(\tau)$ is by showing that 
$\sPr$ is unification of types $\tau$ and $\tauPr_s$ (fresh variant of $\tau_s$), 
since $\mu = \MGU{\tau}{\tauPr_s}$ (and therefore more general then an ordinary unification).
\red{todo do general notions dát stručně vlastnosti unifikací a vstah k MGU}

We show this by constructing the unification $\nu$ (of $\tau$ and $\tauPr_s$) from substitutions $\sPr$ and $\theta$ 
(since $\sPr(\tau) = \theta(\tau_s)$). First we construct $\thPr$ such that $\thPr(\tauPr_s) = \Th(\tau_s) = \sPr(\tau)$.

\red{todo diagram}

Since $\rho = \fresh{\tau}{\tau_s}$ is renaming we can use its inverse $\rho^{-1}$ to translate variables of $\tauPr_s$ back to variables of $\tau_s$. Thus $\thPr \defi \theta \circ \rho^{-1}$ satisfies our needs, because $\theta(\rho^{-1}(\tauPr_s)) = \theta(\tau_s) = \sPr(\tau)$. 

We can merge substitutions $\restrict{\sPr}{\tau}$ with $\thPr$ since their variables cannot clash. 
Let us define $\nu \defi \restrict{\sPr}{\tau} \discup \thPr$.
$\nu(\tau) = \sPr(\tau) = \theta(\tau_s) = \thPr(\tauPr_s) = \nu(\tauPr_s)$, 
thus $\nu$ is unification of $\tau$ and $\tauPr_s$, and therefore $\sPr(\tau) \tord \mu(\tau)$.
We have proven proposition \ref{eq:mostGenSub} for $k = 1$. 
Let us continue with proof of $\trees{k}{\tau} \subseteq \e{k}{\tau}$.

Let $k > 1$.

Let $M \in \trees{k}{\tau}$, we shall show that $M \in \e{k}{\tau}$.

$(M,\s) \in \TS{k}{\tau} = \bigcup_{i=1}^{k-1}  \TSij{i}{k-i}{\tau}$.

Therefore there must be $i,j > 0$ where $i + j = k$ such that $(M,\s) \in \TSij{i}{j}{\tau}$.

Thus $(M,\s) = (\FX,\restrict{\sX \circ \sF}{\tau})$, where

$\alpha = \newVar{\tau}$,
$(F,\sF) \in \TS{i}{\alpha \ar \tau}$,
$(X,\sX) \in \TS{j}{\sF(\alpha)}$.

Due to \IH of proposition \ref{eq:treesEqE}, 
$(F,\sF) \in \TS{i}{\alpha \ar \tau}$ implies $F \in \e{i}{\alpha \ar \tau}$.
Similarly, $(X,\sX) \in \TS{j}{\sF(\alpha)}$ implies $X \in \e{j}{\sF(\alpha)}$. 
Therefore $\abs{F} = i$ and $\abs{X} = j$.

Due to \IH of proposition \ref{eq:mostGenSub}, 
$(F,\sF) \in \TS{i}{\alpha \ar \tau}$ implies $F : \sF(\alpha \ar \tau) = \sF(\alpha) \ar \sF(\tau)$.
Similarly, $(X,\sX) \in \TS{j}{\sF(\alpha)}$ implies $X : \sX(\sF(\alpha))$.
Therefore we can use \mguMp in the following way:

~

$\triRule{F : \sF(\alpha) \ar \sF(\tau)}{X : \sX(\sF(\alpha))}{\sX \in \MGU{\sF(\alpha)}{\sX(\sF(\alpha))}}{\ap{F}{X} : \sX(\sF(\tau))}$

~

\red{todo okomentovat $\sX \in \MGU{\sF(\alpha)}{\sX(\sF(\alpha))}$ a referencnout pozorování že MGU je indempotentní.}

Since $\abs{\FX} = \abs{F} + \abs{X} = i + j = k$, 

and $\FX : \sX(\sF(\tau)) = \restrict{(\sX \circ \sF)}{\tau}(\tau)$,

we have shown that $M \in \e{k}{\tau}$.

Now we follow with proof of proposition \ref{eq:mostGenSub} for $k > 1$.

Let $(\FX,\restrict{(\sX \circ \sF)}{\tau}) \in \TS{k}{\tau}$.

Again, the first part of consequent is proven above and we must prove the second part.
Let us assume that we have substitution $\sPr$ such that 
statement $\FX : \sPr(\tau)$ holds. We shall show that $\sPr(\tau) \tord (\restrict{(\sX \circ \sF)}{\tau})(\tau)$.




%thus $M = s, (s,\tau_s) \in \Gamma, \mu = \MGU{\tau}{\tauPr_s}$,




\red{todo}
\end{proof}

\end{proposition}



\section{Most General Inhabitators}

Asi trochu zbytečná obklika, ale možná se bude pak hodit.. Plus z tho vycházej některý uvahy použitý výše.

\begin{definition}
\begin{align*}
\tauPr \tord  \tau   &\defe \E{\sigma}{\tauPr = \sigma(\tau)} \\
\tauPr \stord \tau   &\defe \tauPr \tord \tau \wedge \neg (\tau \tord \tauPr) \\
\tauPr \nsucc \tau &\ \equiv \tauPr \tord \tau \vee  \neg ( \tau \tord \tauPr)   \\   
\\
\tau_1 \tek      \tau_2  &\defe  (\tau_1 \tord \tau_2) \wedge (\tau_2 \tord \tau_1) \\
\tau_1 \perp     \tau_2  &\defe  (\tau_1 \tord \tau_2) \vee (\tau_2 \tord \tau_1) \\
\tau_1 \parallel \tau_2  &\defe  \neg (\tau_1 \perp  \tau_2) \\
\\
\sigmaPr \square_\tau \sigma   &\defe   \sigmaPr(\tau)\ \square\ \sigma(\tau)\\
%\sigmaPr \ordt   \sigma  &\defe  \sigmaPr(\tau) \tord \sigma(\tau)\\
\text{E.g.:~~} \sigmaPr \nsucct \sigma  &\defe  \sigmaPr(\tau) \nsucc \sigma(\tau)\\
%\sigmaPr \ekt    \sigma  &\defe  \sigmaPr(\tau) \tek \sigma(\tau) \\
\\
\Utt &\defi \{ \s \mid \s(\tau) = \s(\tauPr) \}\\
\MGUtt &\defi \{ \s \in \Utt \mid \A{\sPr \in \Utt}{\sPr \nsucct \s}  \}\\
\\
\It &\defi \{ \sigma \mid \E{M}{M : \sigma(\tau)}  \} \\
\MGI{\tau} &\defi \{ \sigma \in \It \mid \Ain{\sigmaPr}{\It}{\sigmaPr \nsucct \sigma}  \}\\
\end{align*}
\end{definition}


U is for Unificator. I is for Inhabitator. MG is for Most General. 

In general a set of Most General elements can be stated as:
$$\op{MG}(X,\leq) \defi \{ x \in X \mid \Ain{\xPr}{X}{\xPr \leq x \vee \neg(x \leq \xPr)} \}$$

Therefore 
$$\MGUtt = \op{MG}(\Utt,\ordt)$$
$$\MGIt  = \op{MG}(\It, \ordt)$$

~

\begin{lemma}
Let $\sigma \in \MGIt$, $\sigmaPr \in \It$ such that $\sigma \ordt \sigmaPr$,
then $\sigmaPr \ekt \sigma$.
\end{lemma}
\begin{proof}
$\sigmaPr \nsucct \sigma$, since $\sigmaPr(\tau) \nsucc \sigma(\tau)$, from definition of $\MGIt$.\\
Thus $\sigmaPr \ordt \sigma \vee \neg( \sigma \ordt \sigmaPr )$.
Thus $\sigmaPr \ordt \sigma$.
Therefore $\sigmaPr \ekt \sigma$.
\end{proof}

\begin{lemma}
Let $\sigma_1, \sigma_2 \in \MGIt$ such that $\sigma_1 \nekt \sigma_2$,
then $\sigma_1 \parallel_\tau \sigma_2$.  
\end{lemma}
\begin{proof}
$\sigma_1 \nsucct \sigma_2$ and $\sigma_2 \nsucct \sigma_1$ , 
since $\sigma_1, \sigma_2 \in \MGIt$.

$\s_1 \nsucct \s_2 \iff \s_1 \ordt \s_2 \vee \neg(\s_2 \ordt \s_1)$,~~~~~ \textit{(a)}

$\s_2 \nsucct \s_1 \iff \s_2 \ordt \s_1 \vee \neg(\s_1 \ordt \s_2)$,~~~~~ \textit{(b)}

$\s_1 \nekt \s_2 \iff \neg(\s_1 \ordt \s_2) \vee \neg(\s_2 \ordt \s_1)$. \textit{(c)}

\textit{(a)~$\wedge$~(c)~} $\Longrightarrow \neg(\sigma_2 \ordt \sigma_1)$

\textit{(b)~$\wedge$~(c)~} $\Longrightarrow \neg(\sigma_1 \ordt \sigma_2)$

Therefore $\neg(\sigma_1 \ordt \sigma_2) \wedge \neg(\sigma_2 \ordt \sigma_1)$.
\end{proof}




\begin{definition}
A substitution $\rho$ is called a renaming, if it is a permutation on the set of all variables.
\end{definition}

\begin{lemma}
$\tauPr \tek \tau \iff$ $\E{\text{ renaming } \rho}{\tauPr = \rho(\tau)}$.
\end{lemma}
\begin{proof}
From lemmas \ref{lem:ren1} and \ref{lem:ren2}.
\end{proof}


%\begin{lemma} BLBOST
%If $\tauPr \tord \tau_1, \tau_2$, then there is $\s$ such that $\s(\tau_1) = \s(\tau_2) = \tauPr$.
%\end{lemma}
%\begin{proof} 
%$\tauPr \tord \tau_1$ via $\s_1$, $\tauPr \tord \tau_2$ via $\s_2$.
%$\s$ is constructed as $\s = \sPr_1 \cup \sPr_2$, where $\sPr_1, \sPr_2$ are modifications of $\s_1,\s_2$ with renamed (domain) variables.
%Construct $\rho_1, \rho_2$ renaming variables of $\tau_1, \tau_2$ to completely new variables to ensure  that $\op{vars}(\rho_1(\tau_1)) \cap \op{vars}(\rho_2(\tau_2)) = \emptyset$.
%Let $\sPr_1, \sPr_2$
%\end{proof}


\begin{lemma}
\label{lem:muInIt}
Let $(s,\tau_s) \in \Gamma$. Let $\s$ be substitution. Let $\mu \in \U{\tau}{\s(\tau_s)}$.\\ 
Then $\mu \in \It$.  
\end{lemma}
\begin{proof}
$\mu \circ \s$ is substitution, 
thus from \subAx rule we have $s : \mu(\s(\tau_s))$.
$\mu(\tau) = \mu(\s(\tau_s))$, therefore $s : \mu(\tau)$. 

\red{Zkusit pak zobecnit na původní silnější důsledek: $\mu \in \MGIt$. }
\end{proof}


\begin{lemma}
Let $\sigma_1, \sigma_2 \in \MGIt$.
If there is symbol $s$ in $\Gamma$ such that
$s : \s_1(\tau) $ and $ s : \s_2(\tau)$, then
$\s_1 \ekt \s_2$.  
\end{lemma}
\begin{proof}~

Let $(s,\tau_s) \in \Gamma$. 
Let $i \in \{1,2\}$.

Statements $s : \s_i(\tau) $ must be produced by \subAx rule. 

Thus there must be $\Th_i$ such that 

$\tau_i = \s_i(\tau) = \Th_i(\tau_s)$.

Construct renaming $\rho$ variables of $\tau_s$ to completely new variables,

to ensure that $\op{vars}(\tau) \cap \op{vars}(\rho(\tau_s)) = \emptyset$. 

Let $\tauPr_s = \rho(\tau_s)$.

Let $\thPr_i = \Th_i \circ \rho^{-1}$.

%$\tau_i \tord \tauPr_s$ via $\thPr_i$
$\tau_i = \thPr_i(\tauPr_s)$.

We can define substitution $\upsilon_i = \s_i \cup \thPr_i$, because their domains do not clash. 

Furthermore $\tau_i = \upsilon_i(\tau) = \upsilon_i(\tauPr_s)$.

Thus $\upsilon_i \in \U{\tau}{\tauPr_s}$.

Therefore there is $\mu \in \MGU{\tau}{\tauPr_s}$.

Thus $\upsilon_i \nsucct \mu$.

$\upsilon_i \nsucct \mu \iff \upsilon_i \ordt \mu \vee \neg(\mu \ordt \upsilon_i)$.


Furthermore \red{$\upsilon_i \ordt \mu$}, since MGU is unique up to $\ekt$. \red{TODO pořádně}

And from lemma \ref{lem:muInIt} we have $\mu \in \It$.

But $\sigma_i \in \MGIt$, thus $\mu \nsucct \sigma_i$.

$\mu \nsucct \s_i \iff \mu \ordt \s_i \vee \neg(\s_i \ordt \mu)$.

From $\upsilon_i \ordt \mu$, we have $\upsilon_i(\tau) \tord \mu(\tau)$, 
but $\upsilon_i(\tau) = \s_i(\tau)$,
thus $\s_i \ordt \mu$.

So from $\mu \nsucct \s_i$ and from $\s_i \ordt \mu$ we have $\mu \ordt \s_i$.

Now we have both $\mu \ordt \s_i$ and $\s_i \ordt \mu$, thus $\s_i \ekt \mu$.

Finally $\s_1 \ekt \mu \ekt \s_2$.

~

\red{Předělat znění tvrzení a následně i důkazu, protože ten důkaz dokazuje obecnější věc, která se navíc bude velice hodit u důkazu správnosti algoritmu. Staré znění pak formulovat jako důsledek. Nové znění asi něco jako učesanější verze následujícího:}

Let $\sigma \in \MGIt$.
If there is symbol $s$ in $\Gamma$ such that
$s : \s(\tau) $, then $\s \in \MGU{\tau}{\tauPr_s}$.  

~

\red{Formalnějc okecat definici $\upsilon_i$ a ošetřit zbytečně velký domény $\s_i$ - to jde tak že vemem v tý definici $\s_i$ (a klidně i $\thPr_i$) restriktlý na správný proměnný. Druhá věc co se dá rozvést/je dobrý ji referencnou, je to s tim že MGU is unique up to $\ekt$.}
\end{proof}


\begin{lemma}
Let $\sigma_1, \sigma_2 \in \MGIt$.
If there is term $\ap{F}{X}$ such that
$\ap{F}{X} : \s_1(\tau) $ and $\ap{F}{X} : \s_2(\tau)$, then
$\s_1 \ekt \s_2$.  
\end{lemma}
\begin{proof}~

\red{TODO now!}

\end{proof}


\begin{conjecture}
Let $\sigma_1, \sigma_2 \in \MGIt$ such that $\sigma_1 \nekt \sigma_2$. \\
Then $\A{M_1,M_2}{ M_1 : \sigma_1(\tau) \wedge M_2 : \sigma_2(\tau) \then M_1 \neq M_2}$.  
\end{conjecture}
\begin{proof}
By induction on complexity of terms $M_1,M_2$.

\red{TODO case lemmas..}
\end{proof}



~

~

\section{Notes}

\begin{lemma}
\label{lem:ren1}
If $\tauPr \tek \tau$, then there is a \textit{renaming} $\rho$ such that $\tauPr = \rho(\tau)$.
\end{lemma}
\begin{proof}
\red{Roughly:}
$\tauPr \tord \tau \wedge \tau \tord \tauPr$, thus $\E{\rho_1, \rho_2}{\tauPr = \rho_1(\tau), \tau = \rho_2(\tauPr)}$.
$\tauPr = \rho_1(\rho_2(\tauPr))$, $\tau = \rho_2(\rho_1(\tau))$.
Take $\rho_a$ restriction of $\rho_1$ on vars of $\tau$,
Take $\rho_b$ restriction of $\rho_2$ on vars of $\tauPr$.
Thus $\rho_a \circ \rho_b = \rho_b \circ \rho_a = \ids$ 
\red{Todo pořádně..}
\end{proof}

\begin{lemma}
\label{lem:ren2}
If there is a \textit{renaming} $\rho$ such that $\tauPr = \rho(\tau)$, then $\tauPr \tek \tau$.
\end{lemma}
\begin{proof}
From $\tauPr = \rho(\tau)$ follows $\tauPr \tord \tau$. We show that $\tau \tord \tauPr$ by proving $\tau = \rho^{-1}(\tauPr)$. Since $\rho$ is renaming it is a permutation, therefore $\rho$ has inverse and $\rho$ is injective. $\rho \circ \rho^{-1} = \ids$, thus $\rho(\rho^{-1}(\tauPr)) = \tauPr = \rho(\tau)$. From injectivity we have $\rho^{-1}(\tauPr) = \tau$.
\end{proof}



\begin{definition}
$\ids = \{\}$ is the \textit{identity} substitution. 
\end{definition}
\begin{remark}
$\sigma \ordt \ids$ for any $\sigma$, since $\sigma(\tau) \tord \tau$ via $\sigma$, because $\sigma(\tau) = \sigma(\tau)$.
\end{remark}

~

More detailed analysis of $\nsucc$:
\begin{align*}
\tauPr \nsucc \tau &\equiv \neg (\tau \stord \tauPr)   \\
  &\equiv \neg (\tau \tord \tauPr \wedge \neg (\tauPr \tord \tau) )   \\
  &\equiv \neg (\tau \tord \tauPr) \vee \tauPr \tord \tau   \\  
  &\equiv \tauPr \tord \tau \vee  \neg (\tau \tord \tauPr)    \\
  &\equiv \tau \tord \tauPr \then \tauPr \tord \tau   \\  
  &\equiv \tau \tord \tauPr \then \tauPr \tek \tau   \\ 
  &\equiv \tauPr \succcurlyeq \tau \then \tauPr \tek \tau   \\ 
\end{align*}

~

Not so much used predicate definitions:
\begin{align*}
\op{I}_\tau(\sigma)  &\defe \E{M}{M : \sigma(\tau)}  \\
\MGI{\tau}(\sigma) &\defe \op{I}_\tau(\sigma) \wedge \A{\sigmaPr}{(\op{I}_\tau(\sigmaPr) \then\sigmaPr(\tau) \nsucc \sigma(\tau))}\\
\end{align*}


~

~

\section{Probably abandoned stuff}

\newcommand{\subs}[2]{\op{subs}_{#2}(#1)}

\begin{preDefinition}
$ \subs{\tau_g}{n} \defi \{ \sigma_g \mid \E{M}{M : \sigma_g(\tau_g)}, \abs{M} = n \}$
\end{preDefinition}

Problem s touhle starou definicí: asi zahrnuje i zbytečně konkrétní substituce.

Kdybych měl $\Gamma = \{id : \alpha \ar \alpha \}$

Tak $\{\alpha \mapsto Int, \beta \mapsto  Int\} \in \subs{\alpha \ar \beta}{1}$

Čili chceme něco jako Most General Subs který pokrejvá všecky Mka.. 



~










$ \op{weakSubs}(\tau_g) \defi \{ \sigma_g \mid \inhab{\sigma_g(\tau_g)} \}$


\begin{preLemma} 
$ \subs{\tau_g}{1} = \{ \sigma_g \mid (s,\tau_s) \in \Gamma, \sigma_g = \MGU{\tau_g}{\tau^\mathit{fresh}_s}  \} $
\end{preLemma} 




\backmatter
\end{document}

